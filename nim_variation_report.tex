\documentclass{article}
\usepackage{tikz}
\usepackage{graphicx}
\usepackage{amsmath}
\usepackage{fancyhdr}
\usepackage{array}
\usetikzlibrary{positioning,shapes}
\graphicspath{{./report_images/.}}
\pagestyle{fancy}

\begin{document}

\begin{titlepage}
\title{A Variation of NIM: A Report}
\author{Derik Dreher, Sofia Jones, Isayha Raposo}
\maketitle
\thispagestyle{empty}
\end{titlepage}

\lhead{Derik Dreher, Sofia Jones, Isayha Raposo}
\rhead{\today}

\section*{The Solution}
\subsection*{Algorithm Overview}
The CPU has two levels, basic and advanced. In the case the advanced CPU is unable to make a move the basic CPU will be used. If neither the basic or advanced CPU is able to make a move a random move will be made.
Algorithm :
\begin{verbatim}
  get num of piles, pile counts and turn
  
  while game ! over:
    if humans turn:
      get human input
      
    if cpus turn: 
      choice = minimax
      convert minimax result, choice, to move 
      
    check 3 conditions to end game:
     if all piles empty:
       game over = true
     elif 3 piles each with 2 objects remaining & all other piles are empty:
       game over = true
     elif 1 pile with 1 object & 1 other pile with 2 objects &  a third pile with 3 objects & all other piles are empty
       game over = true
       
    if game over & cpus turn:
       Human wins
    else:
      CPU wins
\end{verbatim}

Minimax algorithm for better cpu:
\begin{verbatim}
    choices = []
    if cpus turn: 
      generate children
      for child in children:
        if child, turn in cache:
          get cache
        else: 
          run minimax next iteration
          update cache
         if new choice found:
           add choice
      return max result
     else:
       if new choice found:
         add choice
      return minimum result
\end{verbatim}


\subsection*{Algorithm Correctness}

\subsection*{Data Structures Employed}
\begin{enumerate}
  \item Tree (Game Tree): To choose optimal move a game tree is used, this tree searches for the best move using our \textit{minimax} algorithm. 
  \item Hash map(dictionary): the constraints or blacklist is stored as a hash map. The minimax algorithm also uses cache stored as a dictionary. 
  \item Arrays: for choices and piles
\end{enumerate}
\subsection*{Algorithm Complexity Analysis}
O(b^m)
maximum depth sum of piles
\end{document}


